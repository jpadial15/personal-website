%
% Academic CV - Jorge R. Padial Doble
% ===================================
%
% Formal academic curriculum vitae for graduate degree submission
%

\documentclass[11pt,a4paper]{article}

% Set the page margins - slightly narrower for better space utilization
\usepackage[a4paper,margin=0.85in]{geometry}

% Professional font - Palatino
\usepackage{mathpazo}
\usepackage[T1]{fontenc}
\usepackage{microtype} % Improved typography and letter spacing
\usepackage{xcolor} % For subtle color accents

% Define professional colors
\definecolor{sectioncolor}{RGB}{20, 40, 80} % Dark navy blue
\definecolor{subtletext}{RGB}{100, 100, 100} % Gray for subtle text

% Setup the language
\usepackage[english]{babel}
\hyphenation{Some-long-word}

% Makes resume-specific commands available
\usepackage{resume}
\usepackage{fontawesome}
\usepackage{titlesec}
\usepackage{enumitem}
\usepackage{amsmath}
\usepackage{hyperref}

% Hyperlink setup - subtle and professional
\hypersetup{
    colorlinks=true,
    linkcolor=black,
    urlcolor=sectioncolor,  % Use section color for links
    citecolor=black,
    pdfborder={0 0 0}       % Remove boxes around links
}

% Section formatting - simple and clean with color accent
\titleformat{\section}{\color{sectioncolor}\normalfont\Large\bfseries}{}{0em}{}[\textcolor{sectioncolor}{\titlerule}]

% Subsection formatting - visual hierarchy
\titleformat{\subsection}{\vspace{0.5em}\normalfont\large\itshape}{}{0em}{}
\titlespacing*{\subsection}{0pt}{1em}{0.3em}

% Spacing - optimized for readability
\setlength{\parskip}{0.2em}
\setlength{\parindent}{0em}
\linespread{1.05} % Slightly tighter line spacing for Palatino

\begin{document}

% Header - Traditional format with enhanced typography
\begin{center}
    {\Huge\textbf{JORGE R. PADIAL DOBLE}}\\[0.4cm]
    {\large\textsc{Department of Physics and Astronomy, Vanderbilt University}}\\
    {\normalsize Nashville, TN 37235}\\[0.3cm]
    {\small
    Email: \href{mailto:jorge.r.padial.doble@vanderbilt.edu}{jorge.r.padial.doble@vanderbilt.edu}\\
    GitHub: \href{https://github.com/jpadial15}{github.com/jpadial15}
    }\\[0.2cm]
    \rule{0.7\textwidth}{0.4pt}
\end{center}

\spacedhrule{0.5em}{-0.4em}

%%%%%%%%%%%%%%%%%%%%%%%%%%%%%%%%%%%%%%%%%%%%
% EDUCATION SECTION
%%%%%%%%%%%%%%%%%%%%%%%%%%%%%%%%%%%%%%%%%%%%

\roottitle{Education}

\headedsection
{\textbf{Vanderbilt University}}
{\textsc{Nashville, TN}}
{%
    \headedsubsection
    {Ph.D. in Physics (Astrophysics)}
    {August 2019–December 2025 {\small\textcolor{gray}{(Expected)}}}
    {\bodytext{
        \textbf{Dissertation:} Designed and implemented a robust ETL pipeline ran on 100's terabytes of images, solving the 50-year-old problem of precise solar flare localization from first principles.\\[0.1cm]
        \textbf{Committee Chair:} Kelly Holley-Bockelmann, Ph.D.\\[0.1cm]
        \textbf{Committee Members:} Jesse Spencer-Smith, Ph.D.; Jon Bird, Ph.D.; Karan Jani, Ph.D.\\[0.1cm]
    }}
}

\headedsection
{\textbf{Fisk University}}
{\textsc{Nashville, TN}}
{%
    \headedsubsection
    {Master of Science in Physics; Magna Cum Laude}
    {August 2017–August 2019}
    {\bodytext{
        \textbf{Thesis:} A Hard X-Ray look at Supernova Remnant RCW-86
    }}
}

\headedsection
{\textbf{University of Puerto Rico -- Río Piedras}}
{\textsc{San Juan, PR}}
{%
    \headedsubsection
    {Bachelor of Science in Theoretical and Mathematical Physics; Cum Laude}
    {August 2013–May 2017}
    {}
}

\headedsection
{\textbf{University of Puerto Rico -- Río Piedras}}
{\textsc{San Juan, PR}}
{%
    \headedsubsection
    {Bachelor of Science in Molecular Biology}
    {August 2007–May 2012}
    {}
}

\spacedhrule{0.5em}{-0.4em}

%%%%%%%%%%%%%%%%%%%%%%%%%%%%%%%%%%%%%%%%%%%%
% RESEARCH EXPERIENCE
%%%%%%%%%%%%%%%%%%%%%%%%%%%%%%%%%%%%%%%%%%%%

\roottitle{Research Experience}

\headedsection
{\textbf{Vanderbilt University}}
{\textsc{Nashville, TN}}
{%
    \headedsubsection
    {Graduate Research Assistant}
    {May 2020–Present}
    {\bodytext{
        \textbf{Project:} When and Where is that Flare? A 10-Year Solar Flare Catalogue\\
        Developed and optimized a Python-based ETL pipeline to efficiently extract, clean, and process more than 100 TB of data from AIA, SXI, and XRS, producing a comprehensive solar flare catalog with machine learning insights.\\
        \textbf{Advisor:} Kelly Holley-Bockelmann, Ph.D.
    }}
}

\headedsection
{\textbf{University of Chicago, Department of Computer Science}}
{\textsc{Chicago, IL}}
{%
    \headedsubsection
    {Visiting Researcher}
    {May 2021–Present}
    {\bodytext{
        Collaborated on large-scale data processing and machine learning applications for solar physics research. Utilized high-performance computing resources at Argonne National Laboratory.
    }}
}

\headedsection
{\textbf{Harvard-Smithsonian Center for Astrophysics}}
{\textsc{Cambridge, MA}}
{%
    \headedsubsection
    {Graduate Research Assistant}
    {May 2018–May 2020}
    {\bodytext{
        \textbf{Project:} A Hard X-ray look at the South West Region of RCW-86\\
        Engineered a Python ETL pipeline to integrate and analyze 19 disparate datasets, applying advanced statistical models to uncover insights into non-thermal photon emissions from Supernova Remnants.\\
        \textbf{Advisor:} Smithsonian Astrophysical Observatory Fellowship Program
    }}
}

\headedsection
{\textbf{Arecibo Legacy Fast ALFA Survey Team}}
{\textsc{National Astronomy and Ionosphere Center}}
{%
    \headedsubsection
    {Undergraduate Researcher}
    {January 2014–May 2014}
    {}
    
    \headedsubsection
    {Project A2899 Observing Team}
    {November 2014}
    {\bodytext{
        Four observing nights at Arecibo Observatory
    }}
    
    \headedsubsection
    {A WISE view of Almost Dark ALFALFA Galaxies}
    {June 2014}
    {\bodytext{
        Analyzed Dark Galaxy Candidate AGC-229101 with radio observations from the Arecibo Observatory and infrared data from WISE and Spitzer.
    }}
    
    \headedsubsection
    {Project A2811 Observing Team}
    {November 2013}
    {\bodytext{
        Four observing nights. Used L-Band Wide receivers to search for galaxies of unknown redshift with an optical counterpart galaxies with a $5 < \text{S/N} < 6.5$.
    }}
}

\spacedhrule{0.5em}{-0.4em}

%%%%%%%%%%%%%%%%%%%%%%%%%%%%%%%%%%%%%%%%%%%%
% PUBLICATIONS
%%%%%%%%%%%%%%%%%%%%%%%%%%%%%%%%%%%%%%%%%%%%

\roottitle{Publications}

\subsection*{Peer-Reviewed Publications}

\begin{enumerate}[leftmargin=2em, itemsep=0.8em, parsep=0pt, label=\textbf{[\arabic*]}]
    \item \textbf{Padial-Doble, J. R.}, \& Holley-Bockelmann, K. (2025). ALEXIS: Recreating the X-Ray Emission from the Full Solar Disk as a Linear Combination of Discrete Regions in the Extreme Ultraviolet and Soft X-Rays. \textit{The Astrophysical Journal Supplement Series}, 278, 14. \href{https://doi.org/10.3847/1538-4365/adbe35}{doi:10.3847/1538-4365/adbe35}
\end{enumerate}

\subsection*{Publications in Preparation}

\begin{enumerate}[leftmargin=2em, itemsep=0.8em, parsep=0pt, label=\textbf{[\arabic*]}]
    \item \textbf{Padial-Doble, J. R.}, et al. ALEXIS: Scaling to 100 TB of AIA images and Millions of CPU hours at National Compute Laboratories. In preparation for submission. Expected December 2025.
\end{enumerate}

\spacedhrule{0.5em}{-0.4em}

%%%%%%%%%%%%%%%%%%%%%%%%%%%%%%%%%%%%%%%%%%%%
% PRESENTATIONS
%%%%%%%%%%%%%%%%%%%%%%%%%%%%%%%%%%%%%%%%%%%%

\roottitle{Presentations and Conference Participation}

\subsection*{Invited Talks}

\headedsection
{\textbf{American Meteorological Society}}
{\textsc{New Orleans, LA}}
{%
    \headedsubsection
    {\textit{Invited Talk:} The Automatically Labelled EUV and XRay Incident Solarflare Catalog: Using Big Data and ML Techniques to Pin-Point Solar Flare Locations}
    {January 12–16, 2025}
    {}
}

\headedsection
{\textbf{Data, Analysis, and Software in Heliophysics (DASH) Conference}}
{\textsc{Madrid, Spain}}
{%
    \headedsubsection
    {\textit{Invited Talk:} Automatically Labelled EUV and Xray Incident Solarflare (ALEXIS) catalog for Solar Flare Prediction Machine Learning Models}
    {October 14–18, 2024}
    {\bodytext{Travel expenses covered by NASA.}}
}

\headedsection
{\textbf{University of Michigan Center for All-Clear SEP Forecast}}
{\textsc{Ann Arbor, MI}}
{%
    \headedsubsection
    {\textit{Invited Talk:} The ALEXIS solar flare catalog: Presented results derived from an automated ETL pipeline designed for large-scale solar data processing}
    {August 10, 2024}
    {\bodytext{Travel expenses covered by University of Michigan.}}
}

\headedsection
{\textbf{Space Weather Prediction Center Seminar Series}}
{\textsc{Boulder, CO}}
{%
    \headedsubsection
    {\textit{Invited Talk:} Automatically Labelled EUV and XRay Incident Solarflare (ALEXIS) Catalog}
    {April 4, 2024}
    {}
}

\headedsection
{\textbf{Institute for Theory and Computation, Harvard-Smithsonian CfA}}
{\textsc{Boston, MA}}
{%
    \headedsubsection
    {\textit{Invited Talk:} Automatically Labelled EUV and XRay Incident Solarflare Catalog for Machine Learning}
    {February 2024}
    {\bodytext{\href{https://www.youtube.com/watch?v=1WUj-C2-xUM&t=1866s}{Video available online}}}
}

\headedsection
{\textbf{AstroAI, Harvard-Smithsonian Center for Astrophysics}}
{\textsc{Boston, MA}}
{%
    \headedsubsection
    {\textit{Invited Talk:} Automatically Labelled EUV and XRay Incident Solarflare Catalog for Machine Learning}
    {February 2024}
    {\bodytext{\href{https://www.youtube.com/watch?v=bRWKM0OjYsg&t=1590s}{Video available online}}}
}

\subsection*{Conference Presentations}

\headedsection
{\textbf{SHINE Workshop}}
{\textsc{Juneau, AK}}
{%
    \headedsubsection
    {Poster: Automatically Labelled EUV and XRay Incident Solarflares (ALEXIS Solar Flare Catalog)}
    {August 12–18, 2024}
    {\bodytext{Travel expenses covered by SHINE. \href{https://drive.google.com/file/d/1I8xyoxGDcT0vmq0uRx8S_tFhbHBFnY8t/view}{Poster available online}}}
}

\headedsection
{\textbf{Space Weather Week Workshop}}
{\textsc{Boulder, CO}}
{%
    \headedsubsection
    {Lightning Talk and Poster: Automatically Labelled EUV and XRay Incident Solarflare Catalog for Machine Learning}
    {April 14–19, 2024}
    {\bodytext{Travel expenses covered by NASA. \href{https://drive.google.com/file/d/1g72gxfjo2LF-MlzWOtOsVBnI9WsNb_RH/view}{Poster available online}}}
}

\headedsection
{\textbf{American Geophysical Union -- Winter Meeting}}
{\textsc{San Francisco, CA}}
{%
    \headedsubsection
    {Poster: Automatically Labelled EUV and XRay Incident Solarflare Catalog for Machine Learning: Recreating full disk XRay Emission as a Weighted Linear Combination of Discrete Regions in EUV}
    {December 2023}
    {}
}

\headedsection
{\textbf{University of Chicago AI + Science Summer School}}
{\textsc{Chicago, IL}}
{%
    \headedsubsection
    {Poster: Automatically Labelled EUV and XRay Incident Solarflare Catalog: The need for more data}
    {August 2022}
    {}
}

\headedsection
{\textbf{233rd Meeting of the American Astronomical Society}}
{\textsc{Seattle, WA}}
{%
    \headedsubsection
    {Poster: A Hard X-ray look at the South West Region of RCW-86}
    {January 2019}
    {}
}

\headedsection
{\textbf{Puerto Rico Interdisciplinary STEM Conference}}
{\textsc{UPR-Cayey Campus}}
{%
    \headedsubsection
    {Oral Presentation: L-Band Wide Observations of Star-Forming Galaxies in X-ray Selected Groups: The Quenching of Star Formation in the Group Environment}
    {September 2015}
    {}
}

\subsection*{Journal Clubs and Departmental Seminars}

\headedsection
{\textbf{Vanderbilt Astronomy Journal Club}}
{\textsc{Nashville, TN}}
{%
    \headedsubsection
    {Convex optimization for novel solar flare catalogs}
    {April 2023}
    {}
    
    \headedsubsection
    {How to Train your Flare Prediction Model. Revisiting Robust Sampling of Rare Events -- Ahmadzadeh et al. 2021}
    {April 2021}
    {}
    
    \headedsubsection
    {NuSTAR observes synchrotron emission up to 20 keV located in the South West Region of RCW-86}
    {November 2019}
    {}
}

\subsection*{Workshops and Training}

\headedsection
{\textbf{Arecibo Legacy Fast ALFA Undergraduate Team Workshops}}
{\textsc{National Astronomy and Ionosphere Center}}
{%
    \headedsubsection
    {ALFALFA Workshop}
    {January 2015}
    {\bodytext{
        Three observing nights. Tested a new observing mode developed to search in H$_{\text{I}}$ with a ~85 MHz bandwidth for galaxies of unknown redshift for project A2010.
    }}
    
    \headedsubsection
    {ALFALFA Workshop}
    {January 2014}
    {\bodytext{
        Three observing nights. Student proposal writing leader. Learned about galaxies located in groups and exhibiting tidal interactions together and the importance of H-alpha and R-images.
    }}
}

\spacedhrule{0.5em}{-0.4em}

%%%%%%%%%%%%%%%%%%%%%%%%%%%%%%%%%%%%%%%%%%%%
% HONORS, AWARDS, AND GRANTS
%%%%%%%%%%%%%%%%%%%%%%%%%%%%%%%%%%%%%%%%%%%%

\roottitle{Honors, Awards, and Grants}

\subsection*{Computing Resources}

\begin{itemize}[leftmargin=0em, itemsep=0.5em, label={}]
    \item \textbf{Principal Investigator: ALEXIS Director's Discretionary Allocation} \hfill 2023–Present\\
    Argonne National Laboratory\\
    Awarded 200 TB storage and more than 30,000 node hours
\end{itemize}

\subsection*{Fellowships and Research Support}

\begin{itemize}[leftmargin=0em, itemsep=0.5em, label={}]
    \item \textbf{Graduate Research Assistant Fellowship} \hfill 2019–2025\\
    Vanderbilt University
    
    \item \textbf{Non-degree Visiting Researcher} \hfill 2021–Present\\
    University of Chicago, Department of Computer Science
    
    \item \textbf{Smithsonian Astrophysical Observatory Fellowship Award} \hfill 2018 \& 2019\\
    Fisk University
    
    \item \textbf{Fisk-Vanderbilt Master's to Ph.D. Bridge Fellowship} \hfill 2017–2019\\
    Fisk University and Vanderbilt University\\
    \href{https://www.fisk-vanderbilt-bridge.org/program}{Bridge Program}
\end{itemize}

\subsection*{Honors and Awards}

\begin{itemize}[leftmargin=0em, itemsep=0.5em, label={}]
    \item \textbf{2nd Place, Best Student Presentation} \hfill January 2025\\
    American Meteorological Society (\$150 prize)
    
    \item \textbf{Travel Grant — DASH 2024 Conference} \hfill October 2024\\
    All expenses paid by NASA
    
    \item \textbf{Travel Grant — SHINE 2024 Conference} \hfill August 2024\\
    All expenses paid by SHINE
    
    \item \textbf{Travel Grant — Space Weather Week Workshop 2024} \hfill April 2024\\
    All expenses paid by NASA
    
    \item \textbf{Dean's List} \hfill September 2015–May 2017\\
    University of Puerto Rico — Río Piedras
\end{itemize}

\subsection*{Professional Memberships and Leadership}

\begin{itemize}[leftmargin=0em, itemsep=0.5em, label={}]
    \item \textbf{Member, High Energy Astrophysics Division} \hfill 2017–2019\\
    American Astronomical Society
    
    \item \textbf{President, Society of Physics Students} \hfill January 2016–January 2017\\
    University of Puerto Rico — Río Piedras
    
    \item \textbf{Member, Society of Physics Students} \hfill January 2015–January 2016\\
    University of Puerto Rico — Río Piedras
    
    \item \textbf{Captain, Men's Soccer Team} \hfill August 2007–May 2012\\
    NCAA Division II Soccer Scholarship\\
    University of Puerto Rico — Río Piedras
\end{itemize}

\spacedhrule{0.5em}{-0.4em}

%%%%%%%%%%%%%%%%%%%%%%%%%%%%%%%%%%%%%%%%%%%%
% TEACHING EXPERIENCE
%%%%%%%%%%%%%%%%%%%%%%%%%%%%%%%%%%%%%%%%%%%%

\roottitle{Teaching Experience}

\headedsection
{\textbf{Vanderbilt University, Department of Physics and Astronomy}}
{\textsc{Nashville, TN}}
{%
    \headedsubsection
    {Teaching Assistant, Solar System Astronomy}
    {Winter 2021}
    {}
    
    \headedsubsection
    {Teaching Assistant, Introduction to Astrophysics (Undergraduate)}
    {Fall 2020}
    {}
    
    \headedsubsection
    {Lecturer, Undergraduate Astronomy Laboratory}
    {2019 -- 2020}
    {}
}

\spacedhrule{0.5em}{-0.4em}

%%%%%%%%%%%%%%%%%%%%%%%%%%%%%%%%%%%%%%%%%%%%
% TECHNICAL SKILLS
%%%%%%%%%%%%%%%%%%%%%%%%%%%%%%%%%%%%%%%%%%%%

\roottitle{Technical Skills}

\begin{description}[leftmargin=0em, labelwidth=0em, itemsep=0.5em]
    \item[\textbf{Programming Languages:}] Python, R, SQL, BASH, JavaScript, HTML, CSS
    
    \item[\textbf{Scientific Libraries:}] NumPy, Pandas, Matplotlib, SciPy, Scikit-learn, Scikit-image, TensorFlow, PyTorch, OpenCV, Astropy, SunPy, AIApy, XArray, SQLAlchemy, CVXpy
    
    \item[\textbf{Data Formats \& Databases:}] FITS, HDF5, netCDF4, Parquet, SQLite, PostgreSQL
    
    \item[\textbf{Tools \& Platforms:}] Linux/UNIX, macOS, Windows (WSL2), Git, GitHub, Docker, Kubernetes, Jupyter Notebook, VS Code, \LaTeX, Overleaf
    
    \item[\textbf{Web Development:}] Django, FastAPI, Dash, Streamlit, Plotly, BeautifulSoup
    
    \item[\textbf{High-Performance Computing:}] Supercomputing resources at Argonne National Laboratory (30,000+ node hours), parallel processing, distributed systems
    
    \item[\textbf{Languages:}] English (Native — Written/Spoken/Interpretation), Spanish (Native — Written/Spoken/Interpretation)
\end{description}

\spacedhrule{0.5em}{-0.4em}

%%%%%%%%%%%%%%%%%%%%%%%%%%%%%%%%%%%%%%%%%%%%
% SERVICE AND OUTREACH
%%%%%%%%%%%%%%%%%%%%%%%%%%%%%%%%%%%%%%%%%%%%

\roottitle{Service and Outreach}

\headedsection
{\textbf{Smithsonian National Air and Space Museum}}
{\textsc{Washington, DC}}
{%
    \headedsubsection
    {Bilingual science chat for museum visitors}
    {September 21, 2019}
    {}
}

\headedsection
{\textbf{Bancroft Elementary Saturday School}}
{\textsc{Washington, DC}}
{%
    \headedsubsection
    {Bilingual science chat for 4th and 5th graders}
    {January 13, 2018}
    {}
}

\headedsection
{\textbf{Hume-Fogg Academic High School}}
{\textsc{Nashville, TN}}
{%
    \headedsubsection
    {Science Inclusion Talk}
    {February 9, 2018}
    {}
}

\headedsection
{\textbf{Arecibo Observatory Public Observation Day}}
{\textsc{Arecibo, PR}}
{%
    \headedsubsection
    {Society of Physics Students UPRRP Chapter}
    {May 2015 and July 2015}
    {}
}

\headedsection
{\textbf{International SUN-Day}}
{\textsc{Carolina, PR}}
{%
    \headedsubsection
    {ASPIRA of Puerto Rico}
    {July 2015}
    {}
}

\headedsection
{\textbf{University Gardens School of Mathematics and Science}}
{\textsc{Carolina, PR}}
{%
    \headedsubsection
    {Invited Speaker to Astrophysics Club}
    {July 2015}
    {}
}

\end{document}
